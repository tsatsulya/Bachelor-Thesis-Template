%% Работа с русским языком
\usepackage{cmap}			 % поиск в PDF
\usepackage{mathtext} 		 % русские буквы в формулах
\usepackage[T1,T2A]{fontenc}
\usepackage[utf8]{inputenc}	 % кодировка исходного текста
\usepackage[russian]{babel}	 % локализация и переносы

%% Пакеты для работы с математикой
\usepackage{amsmath,amsfonts,amssymb,amsthm,mathtools}
\usepackage{icomma}

%% Нумерация формул (опционально)
%\mathtoolsset{showonlyrefs=true} % показывать номера только у тех формул, на которые есть \eqref{} в тексте.
%\usepackage{leqno}               % нумерация формул слева

%% Шрифты
\usepackage{euscript}	 % шрифт "Евклид"
\usepackage{mathrsfs}    % красивый мат. шрифт
\usepackage{setspace}
%% Некоторые полезные макросы для дебага (в случае недоверия авторам шаблона)
\makeatletter
\newcommand\thefontsize{The current font size is: \f@size pt} % пример: \section{\thefontsize}
\makeatother

%% Настройка размеров шрифтов
\makeatletter
\setlength{\headheight}{28pt}
%% TODO: мне не удалось разобраться, как грамотно подбирать второе число в
%% \@setfontsize\*, но ряд эксппериментов показывает, что "10" выравнивает текст весьма прилично :)
\renewcommand\Huge{\@setfontsize\Huge{14pt}{10}}
\renewcommand\huge{\@setfontsize\huge{14pt}{10}}
\renewcommand\Large{\@setfontsize\Large{14pt}{10}}
\renewcommand\large{\@setfontsize\large{12pt}{10}}
\makeatother

%% Поля (геометрия страницы)
\usepackage[left=3cm,right=1.5cm,top=3cm,bottom=2cm,bindingoffset=0cm]{geometry}

%% Русские списки
\usepackage{enumitem}
\makeatletter
\AddEnumerateCounter{\asbuk}{\russian@alph}{щ}
\makeatother

%% Работа с картинками
\usepackage{caption}
\captionsetup{justification=centering} % центрирование подписей к картинкам
\usepackage{graphicx}                  % вставки рисунков
\graphicspath{{images/}{images2/}}     % папки с картинками
\setlength\fboxsep{3pt}                % отступ рамки \fbox{} от рисунка
\setlength\fboxrule{1pt}               % толщина линий рамки \fbox{}
\usepackage{wrapfig}                   % обтекание рисунков и таблиц текстом

%% Работа с таблицами
\usepackage{array,tabularx,tabulary,booktabs} % дополнительная работа с таблицами
\usepackage{longtable}                        % длинные таблицы
\usepackage{multirow}                         % слияние строк в таблице

%% Красная строка
\setlength{\parindent}{2em}

%% Интервалы
\linespread{1}
\usepackage{multirow}

%% TikZ
\usepackage{tikz}
\usetikzlibrary{graphs,graphs.standard}

%% Верхний колонтитул
\usepackage{fancyhdr}
\pagestyle{fancy}

%% Перенос знаков в формулах (по Львовскому)
\newcommand*{\hm}[1]{#1\nobreak\discretionary{}{\hbox{$\mathsurround=0pt #1$}}{}}

%% Дополнительно
\usepackage{float}   % добавляет возможность работы с командой [H] которая улучшает расположение на странице
\usepackage{gensymb} % красивые градусы
\usepackage{caption} % пакет для подписей к рисункам, в частности, для работы caption*
\usepackage{listings} % пакет для листингов с кодом

% Hyperref (для ссылок внутри  pdf)
\usepackage[unicode, pdftex]{hyperref}

% Отступ перед первым абзацем в каждом разделе
\usepackage{indentfirst}

\usepackage[russian]{babel}
\usepackage{geometry}
\geometry{left=3cm,right=1.5cm,top=2cm,bottom=2cm, headheight=2cm, headsep=0.5cm}

\usepackage{xcolor}

\usepackage{titlesec}

% Настройка размеров (можно использовать конкретные pt)
\titleformat{\section}
  {\normalfont\fontsize{17}{19}\bfseries}{\thesection}{1em}{}

\titleformat{\subsection}
  {\normalfont\fontsize{15}{17}\bfseries}{\thesubsection}{1em}{}

\titleformat{\subsubsection}
  {\normalfont\fontsize{14}{15}\bfseries}{\thesubsubsection}{1em}{}

% Интервалы вокруг заголовков
\titlespacing*{\section}{0pt}{12pt}{6pt}
\titlespacing*{\subsection}{0pt}{10pt}{4pt}
\titlespacing*{\subsubsection}{0pt}{10pt}{4pt}

\definecolor{codegreen}{rgb}{0,0.6,0}
\definecolor{codegray}{rgb}{0.5,0.5,0.5}
\definecolor{codepurple}{rgb}{0.58,0,0.82}


\lstdefinestyle{mystyle}{
    commentstyle=\color{codegreen},
    keywordstyle=\color{magenta},
    numberstyle=\tiny\color{codegray},
    stringstyle=\color{codepurple},
    basicstyle=\ttfamily\footnotesize,
    breakatwhitespace=false,
    breaklines=true,
    captionpos=b,
    keepspaces=true,
    numbers=left,
    numbersep=5pt,
    showspaces=false,
    showstringspaces=false,
    showtabs=false,
    tabsize=2,
    frame=lines,
    inputencoding=utf8,
    extendedchars=\true,
    }


\lstset{style=mystyle}
\newcommand{\code}[1]{\texttt{#1}}

\usetikzlibrary{shapes,arrows,positioning}
\usepackage{lipsum} % Для демонстрации текста


% Определение стиля для TypeScript
\lstdefinelanguage{TypeScript}{
    keywords={let, const, type, interface, class, function, return, if, else, for, while, import, export},
    sensitive=true,
    comment=[l]{//},
    morecomment=[s]{/*}{*/},
    morestring=[b]',
    morestring=[b]",
    alsoletter={-},
    morekeywords=[2]{number, string, boolean, any, void, null, undefined}
}

\lstset{
    basicstyle=\ttfamily\small,
    keywordstyle=\color{blue},
    commentstyle=\color{green!50!black},
    stringstyle=\color{red},
    numbers=left,
    numberstyle=\tiny\color{gray},
    stepnumber=1,
    numbersep=5pt,
    backgroundcolor=\color{white},
    showspaces=false,
    showstringspaces=false,
    showtabs=false,
    tabsize=2,
    captionpos=b,
    breaklines=true,
    breakatwhitespace=true,
    frame=single,
    inputencoding=utf8,
    extendedchars=\true
    }

\usepackage{csquotes}
\usepackage[style=gost-numeric,  % Стиль по ГОСТ (или другой)
            backend=biber,      % Используем Biber вместо BibTeX
            sorting=none,       % Порядок цитирования
            language=auto]{biblatex}
\addbibresource{references.bib}

