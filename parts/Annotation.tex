
\begin{center}
\subsubsection*{Удаление примитивных типов из высокоуровневого языка программирования для улучшения качества семантического анализа}
\textbf{Ермолаева Варвара Дмитриевна}
\end{center}

В современных управляемых языках программирования (таких как Java, C\#, Kotlin) сохраняется исторически сложившееся разделение типов на примитивные (например, \texttt{int}, \texttt{boolean}) и объектные (ссылочные). Это разделение, хотя и направлено на повышение производительности, приводит к:

\begin{itemize}[leftmargin=*,label={--}]
    \item Несогласованности в системе типов
    \item Усложнению семантического анализа
    \item Ограничению возможностей объектно-ориентированного программирования (например, отсутствие полиморфизма для примитивов, необходимость упаковки/распаковки)
\end{itemize}

\medskip

Целью данной работы является улучшение качества семантического анализа за счёт устранения дуализма примитивных и объектных типов в высокоуровневом языке программирования с управляемой средой исполнения. Для этого предложена стратегия <<Примитивы как оптимизация компилятора>>, при которой:

\begin{itemize}[leftmargin=*,label={--}]
    \item Все типы унифицируются в рамках объектной модели
    \item Примитивные операции выполняются непосредственно над объектами (например, \texttt{Int} вместо \texttt{int})
    \item Низкоуровневые оптимизации (замена объектов на примитивы) выполняются на этапе компиляции, не влияя на семантику языка
\end{itemize}

\medskip

В работе проведён:

\begin{itemize}[leftmargin=*,label={--}]
    \item Анализ существующих подходов (Java, C\#, Kotlin)
    \item Разработана формальная спецификация унифицированной системы типов
    \item Реализованы:
    \begin{itemize}
        \item Модификации семантического анализа для поддержки объектных примитивоподобных типов
        \item Алгоритмы свёртки констант и преобразования типов
        \item Механизм оптимизации AST для замены объектов на примитивы
    \end{itemize}
\end{itemize}

\medskip

Результаты показали, что предложенное решение:

\begin{itemize}[leftmargin=*,label={--}]
    \item Упрощает семантический анализ за счёт устранения специальных правил для примитивов
    \item Сохраняет производительность благодаря оптимизациям на этапе компиляции
    \item Повышает удобство разработки, обеспечивая единообразие типов
\end{itemize}

\textbf{Ключевые слова:} система типов, примитивные типы, семантический анализ, оптимизация компилятора, унификация типов.

\newpage