\section{Обзор существующих решений}
\label{sec:Chapter2} \index{Chapter2}

\subsection{Интеграция примитивных типов в высокоуровневую систему типов и их влияние на семантический анализ}

\subsubsection{Стратегия: Явное разделение с автоматическим преобразованием (Autoboxing/Autounboxing)}

\begin{itemize}
    \item \textbf{Суть:} Примитивные типы (`int`, `float`, `boolean`) и объектные типы (`Integer`, `Float`, `Boolean`) существуют параллельно и явно различимы в системе типов языка. Компилятор автоматически вставляет преобразования (боксинг - примитив -> объектная обертка, анбоксинг - объектная обертка -> примитив) там, где контекст требует типа другого вида (например, передача `int` в метод, ожидающий `Object`, или использование `Integer` в арифметической операции).
    \begin{itemize}
        \item Боксинг: примитив $\rightarrow$ объектная обёртка
        \item Анбоксинг: объектная обёртка $\rightarrow$ примитив
    \end{itemize}

    \item \textbf{Влияние на систему типов:}
    \begin{itemize}
        \item Явный дуализм типов
        \item Примитивы не являются частью объектной иерархии
        \item Необходимость учёта обоих "миров" в правилах подтипирования
    \end{itemize}

    \item \textbf{Влияние на семантический анализ:} \textcolor{red}{Значительно усложняет анализ}
    \begin{itemize}
        \item Отслеживание контекстов, требующих преобразований
        \item Разрешение перегрузок методов с разными параметрами
        \item Обработка потенциальных \code{NullPointerException}
        \item Учёт различий в семантике операторов (напр., \code{==})
        \item Отдельные ветви кода для проверки типов
    \end{itemize}

    \item \textbf{Преимущества:}
    \begin{itemize}
        \item Производительность примитивов
        \item Полиморфность объектных обёрток
        \item Понятность для разработчиков
    \end{itemize}

    \item \textbf{Недостатки:}
    \begin{itemize}
        \item Сложная система типов
        \item Риск \code{NullPointerException}
        \item Накладные расходы на преобразования
        \item Концептуальный разрыв
    \end{itemize}

    \item \textbf{Примеры:} Java (классический пример), ранние версии C\#
\end{itemize}

\begin{table}[h]
\centering
\caption{Сравнение поведения операторов}
\begin{tabular}{ll}
\toprule
\textbf{Операция} & \textbf{Семантика} \\
\midrule
\code{==} для примитивов & Сравнение значений \\
\code{==} для объектов & Сравнение ссылок \\
Арифметические операции & Автоматический анбоксинг \\
\bottomrule
\end{tabular}
\end{table}

\begin{lstlisting}[language=java,caption=Пример в Java]
Integer x = 10;  // Автобоксинг
int y = x;       // Автоанбоксинг
if (y == x) {    // Сравнение с автоанбоксингом
    System.out.println("Equal");
}
\end{lstlisting}

\end{document}