\section{Результаты и выводы}
\label{sec:Chapter5} \index{Chapter5}
\subsection{Теоретические достижения}
\begin{itemize}[leftmargin=*]
    \item Разработана формальная спецификация унифицированной системы типов, полностью устраняющая разделение на примитивные и объектные типы
    \item Доказана возможность интеграции примитивных операций в объектную модель без потери производительности
    \item Сформулированы новые правила семантического анализа для унифицированной системы типов
\end{itemize}

\subsection*{Практическая реализация}
\begin{itemize}[leftmargin=*]
    \item Создан механизм преобразования AST, обеспечивающий:
    \begin{itemize}
        \item Поддержку объектных примитивоподобных типов (\texttt{Int}, \texttt{Number}, \texttt{Boolean})
        \item Автоматическую оптимизацию на этапе компиляции
        \item Корректную работу с union-типами и дженериками
    \end{itemize}
    \item Реализованы алгоритмы свёртки констант и проверки типов
    \item Переработана существующая система тестовых сценариев для покрытия всех случаев работы с примитивоподобными объектами
\end{itemize}

\subsection{Оценка эффективности}
\begin{itemize}[leftmargin=*]
    \item Упрощение семантического анализатора:
    \begin{itemize}
        \item Сокращение количества специальных правил ветвлений в коде анализатора\%
        \item Удаление большого объема неэффективного и устаревшего кода\%
    \end{itemize}
    \item Сохранение производительности:
    \begin{itemize}
        \item Устранены накладные расходы на boxing/unboxingустранены
    \end{itemize}
    \item Улучшение опыта разработки:
    \begin{itemize}
        \item Единообразная работа с типами
        \item Уменьшение количества неинтуитивных сценариев со скрытым от разработчика поведением
    \end{itemize}
\end{itemize}

\subsection{Перспективы развития}
\begin{itemize}[leftmargin=*]
    \item Имплементация модуля верификатора для окончательной проверки отсутствия включений примитивных типов в анализаторе
    \item Оптимизация работы с массивами примитивоподобных объектов с помощью изменения рантайма
\end{itemize}

\subsection{Практическая значимость}
\begin{itemize}[leftmargin=*]
    \item Внедрение решения в промышленный компилятор
    \item Снижение порога входа для новых разработчиков
    \item Устранение целого класса ошибок, связанных с преобразованием типов
\end{itemize}

\begin{center}
\fbox{\parbox{0.9\textwidth}{
Таким образом, предложенное решение гарантирует значительное улучшение и упрощение семантического анализа высокоуровневого языка программирования с управляемой средой исполнения посредством избавления от традиционного разделения на примитивы и объекты, сочетая преимущества объектной модели с производительностью низкоуровневых операций.
}}
\end{center}

\newpage
