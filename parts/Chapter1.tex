\newpage
\section{Постановка задачи}
\label{sec:Chapter1} \index{Chapter1}

\subsection{Контекст}
В данной работе речь пойдет о высокоуровневом языке программирования с управляемой средой исполнения, поддерживающей два языка программирования~--- статически типизированный язык, похожий в большей степени на TypeScript, и динамически типизированный, схожий с JavaScript.

Язык поддерживает императивные, объектно-ориентированные, функциональные и шаблонные паттерны программирования и комбинирует разные семантические аспекты TypeScript, Java и Kotlin. На данный момент язык находится в активной стадии разработки.

Примитивные типы в разрабатываемом языке обладают основными характеристиками, описанными ранее~--- они:
\begin{itemize}
    \item не участвуют в подтипировании (в том числе с типом \code{Object})
    \item не могут быть компонентами юнион-типов
    \item не могут приниматься в шаблонных в качестве аргумента
    \item не имеют методов и других свойств объектов
\end{itemize}

Для обеспечения указанной функциональности язык предоставляет объектные аналоги примитивных типов (далее - объектные аналоги примитивных типов, примитивоподобные объекты), представляющие собой объектную обертку с соответствующими методами и правилами наследования (например, \code{Int}~--- объектный аналог примитивного типа \code{int}).
В случаях необходимости между примитивным типом и его аналогом происходит скрытая конверсия, что усложняет семантический анализ программ. Более того, синтаксически допускаются конструкции вида \code{int | undefined} или \code{Set<int>}, которые интерпретируются компилятором как \code{Int | undefined} и \code{Set<Int>}. Несоответствие свойств \code{int} и \code{Int} приводит к непредвиденному поведению в краевых случаях.

Далее в тексте \code{int} и \code{Int} будут использоваться в качестве основного примера.

\vspace{0.5cm}
\noindent\textbf{Цель работы:} улучшение семантического анализа программ посредством устранения концепции примитивных типов как отдельной категории и унификации системы типов.

\subsection{Задачи исследования}
Для достижения поставленной цели необходимо решить следующие задачи:
\begin{enumerate}[leftmargin=*,align=left]
    \item[\textbf{1.}] Провести детальный анализ текущего состояния семантического анализа с целью идентификации мест, где наличие примитивных типов приводит к:
    \begin{itemize}[label={--}]
        \item усложнению логики анализа
        \item неоднозначности интерпретации
        \item необходимости специальных правил обработки (скрытые конверсии, union-типы, generics)
    \end{itemize}

    \item[\textbf{2.}] Разработать формальную спецификацию унифицированной системы типов, которая:
    \begin{itemize}[label={--}]
        \item исключает концепцию примитивных типов
        \item интегрирует их функциональность в объектную модель
        \item сохраняет производительность операций (арифметические, логические)
        \item обеспечивает корректность семантики
    \end{itemize}

    \item[\textbf{3.}] Спроектировать и реализовать модификации в подсистеме семантического анализа, обеспечивающие:
    \begin{itemize}[label={--}]

        \item проверку совместимости типов
        \item разрешение перегрузок
        \item вывод типов
        \item работу с метаданными
    \end{itemize}

    \item[\textbf{4.}] Разработать набор тестовых сценариев, охватывающих:
    \begin{itemize}[label={--}]
        \item шаблонные с числовыми аргументами
        \item union-типы
        \item операции с числовыми значениями
        \item проблемные краевые случаи
    \end{itemize}

    \item[\textbf{5.}] Оценить влияние решения на:
    \begin{itemize}[label={--}]
        \item сложность и точность семантического анализа
        \item удобство разработки
        \item потенциальную производительность
    \end{itemize}
    с использованием формальных метрик:
    \begin{itemize}[label={--}]
        \item уменьшение количества специальных правил
        \item сокращение ветвлений в коде анализатора
        \item результаты тестовых сценариев
    \end{itemize}
\end{enumerate}

\subsection{Требования к решению}
\subsubsection{Функциональные требования}
\begin{enumerate}[leftmargin=*,align=left]
    \item[\textbf{1.}] \textbf{Унификация системы типов:} Разработанное решение должно обеспечить:
    \begin{itemize}[label={--}]
        \item полную интеграцию всех типов (числовых, логических) в общую объектную иерархию
        \item участие базовых типов в механизмах подтипирования
        \item устранение разделения на примитивные и объектные типы
    \end{itemize}

    \item[\textbf{2.}] \textbf{Поддержка продвинутых конструкций:} Все типы должны корректно использоваться в:
    \begin{itemize}[label={--}]
        \item \textbf{Union-типах:} конструкции вида \code{number | undefined} или \code{boolean | null} должны быть валидными
        \item \textbf{Дженериках:} базовые типы как аргументы (\code{Set<number>}, \code{Map<string, boolean>}) без скрытых преобразований
        \item \textbf{Наследовании/Подтипировании:} соблюдение общих правил для базовых типов
    \end{itemize}

    \item[\textbf{3.}] \textbf{Устранение скрытых конверсий:} Необходимо исключить:
    \begin{itemize}[label={--}]
        \item автоматические механизмы упаковки (boxing)
        \item автоматическую распаковку (unboxing) значений
        \item неявные преобразования между разными представлениями типов
    \end{itemize}

    \item[\textbf{4.}] \textbf{Семантическая корректность:} Анализатор должен обеспечивать:
    \begin{itemize}[label={--}]
        \item точное выявление типовых ошибок
        \item однозначное разрешение типов в выражениях
        \item предсказуемое поведение программ
    \end{itemize}

    \item[\textbf{5.}] \textbf{Сохранение синтаксиса:} Требуется:
    \begin{itemize}[label={--}]
        \item сохранить существующий синтаксис (\code{int}, \code{number}, \code{boolean})
        \item обеспечить соответствие внутренней интерпретации унифицированной модели
    \end{itemize}
\end{enumerate}

\subsubsection{Нефункциональные требования}
\begin{enumerate}[leftmargin=*,align=left]
    \item[\textbf{1.}] \textbf{Производительность:}
    \begin{itemize}[label={--}]
        \item отсутствие существенного замедления семантического анализа
        \item оптимизация алгоритмов проверки типов
    \end{itemize}

    \item[\textbf{2.}] \textbf{Эффективность кода:}
    \begin{itemize}[label={--}]
        \item отсутствие заметного снижения производительности исполняемого кода
        \item сравнимая эффективность с оригинальной реализацией
    \end{itemize}

    \item[\textbf{3.}] \textbf{Качество кода анализатора:}
    \begin{itemize}[label={--}]
        \item повышение читаемости и поддерживаемости
        \item модульная структура
        \item упрощение логики обработки типов
    \end{itemize}

    \item[\textbf{4.}] \textbf{Удобство разработки:}
    \begin{itemize}[label={--}]
        \item интуитивно понятная модель типов
        \item снижение когнитивной нагрузки
        \item минимизация неочевидных ошибок
    \end{itemize}
\end{enumerate}

\subsection{Ожидаемые результаты}
В результате выполнения дипломной работы ожидается получить:

\begin{itemize}
    \item \textbf{Спецификацию унифицированной системы типов} для высокоуровневого управляемого языка, включающую:
    \begin{itemize}[label={--}]
        \item формальное описание типовой системы
        \item правила подтипирования
        \item алгоритмы проверки типов
    \end{itemize}

    \item \textbf{Модифицированную подсистему семантического анализа} с:
    \begin{itemize}[label={--}]
        \item поддержкой унифицированных типов
        \item устранением специальных случаев для примитивов
        \item улучшенной архитектурой
    \end{itemize}

    \item \textbf{Комплекс тестовых примеров}, покрывающих:
    \begin{itemize}[label={--}]
        \item базовые операции с типами
        \item union-типы и дженерики
        \item краевые случаи
        \item производительность анализа
    \end{itemize}

    \item \textbf{Оценочные метрики} по:
    \begin{itemize}[label={--}]
        \item сложности семантического анализа
        \item эффективности работы компилятора
        \item удобству использования языка
    \end{itemize}
\end{itemize}

\newpage
